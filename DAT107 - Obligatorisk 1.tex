\documentclass{article}

\usepackage{listings}
\usepackage{xcolor}
\usepackage{color}

\usepackage{parskip}

\usepackage[utf8]{inputenc}
\usepackage[norsk]{babel}
\usepackage[T1]{fontenc}

\definecolor{codegreen}{rgb}{0,0.1,0}
\definecolor{codegray}{rgb}{0.5,0.5,0.5}
\definecolor{codeblue}{HTML}{3875d6}
\definecolor{backcolour}{HTML}{F2F2F2}
\definecolor{bookColor}{cmyk}{0,0,0,0.90}  
\color{bookColor}

\lstset{upquote=true}

\lstdefinestyle{mystyle}{  
	commentstyle=\color{codegreen},
	keywordstyle=\color{codeblue},
	stringstyle=\color{codepurple},
	basicstyle=\footnotesize\ttfamily,
	breakatwhitespace=false,
	breaklines=true,
	captionpos=b,
	keepspaces=true,
	numbersep=10pt,
	showspaces=false,
	showstringspaces=false,
	showtabs=false,
}
\lstset{style=mystyle}

\begin{document}
	
		
\section*{Oppgave}
\textbf{a)} For å oppfylle kravene oppgitt i oppgaven om å lagre informasjon om bilene og eierne, ble det vurdert å lage 5 tabeller. Nedenfor er SQL-koden for å opprette tabellene med forklaring.

Vi oppretter tabellen \textbf{eier} for å lagre informasjon om eierne, inkludert navn, adresse, telefonnummer og e-postadresse. Hver eier får en unik ID som primærnøkkel.

\begin{lstlisting}[language=SQL]
CREATE TABLE IF NOT EXISTS eier (
    id SERIAL PRIMARY KEY,
    navn VARCHAR(255) NOT NULL,
    adresse VARCHAR(255) NOT NULL,
    telefon VARCHAR(20),
    epost VARCHAR(255)
);
\end{lstlisting}

Deretter oppretter vi tabellen \textbf{kjoretoy} for å lagre informasjon om kjøretøyene, som for eksempel registreringsnummer. Hver kjøretøy får en unik ID som primærnøkkel. Tanken med denne tabellen er å kunne lagre flere kjøretøy for hver eier, og dermed håndtere situasjoner der en eier har flere biler.

\begin{lstlisting}[language=SQL]
CREATE TABLE IF NOT EXISTS kjoretoy (
    id SERIAL PRIMARY KEY,
    registreringsnummer VARCHAR(20) UNIQUE NOT NULL
);
\end{lstlisting}

\textbf{kjoretoy\_eier} tabellen bruker vi som en mellomtabell for å hondtere forholdet mellom kjøretøy og eiere, vi tenkte at en eier kan ha flere kjøretøy.
\begin{lstlisting}[language=SQL]
CREATE TABLE IF NOT EXISTS kjoretoy_eier (
    kjoretoyid INTEGER,
    eierid INTEGER,
    FOREIGN KEY (kjoretoyid) REFERENCES kjoretoy(id),
    FOREIGN KEY (eierid) REFERENCES eier(id)
);
\end{lstlisting}


I \textbf{bompassering} tabellen lagrer vi informasjon om hver passering, inkludert tidspunktet for passeringen, kjøretøyets ID og bomstasjonens ID. Vi bruker fremmednøkler for å referere til kjøretøy og bomstasjoner.
\begin{lstlisting}[language=SQL]
CREATE TABLE IF NOT EXISTS bompassering (
    id SERIAL PRIMARY KEY,
    passeringstid TIMESTAMP NOT NULL

    kjoretoyid INTEGER,
    bomstasjonid INTEGER,
    FOREIGN KEY (kjoretoyid) REFERENCES kjoretoy(id),
    FOREIGN KEY (bomstasjonid) REFERENCES bomstasjon(id),
);
\end{lstlisting}

\textbf{bomstasjon} tabellen lagrer informasjon om bomstasjonene, inkludert navn og plassering.
\begin{lstlisting}[language=SQL]
CREATE TABLE IF NOT EXISTS bomstasjon (
    id SERIAL PRIMARY KEY,
    navn VARCHAR(255) NOT NULL,
    plassering VARCHAR(255) NOT NULL
);
\end{lstlisting}


\textbf{c)} Hvis det ikke er behov å lagre telefonnummeret i eier tabellen, så er det mulig å fjerne det på denne måten. 



\begin{lstlisting}[language=SQL]
ALTER TABLE eier DROP COLUMN telefon;
\end{lstlisting}
	
\end{document}
